\section{test}\label{test}

$\{\,x_1,x_2\,\} $

\begin{longtable}[c]{@{}ll@{}}
\toprule\addlinespace
$x$ & $f(x)$
\\\addlinespace
\midrule\endhead
$a$ & $x_1$
\\\addlinespace
$b$ & $x_2$
\\\addlinespace
\bottomrule
\end{longtable}

Ein Graph \emph{g} mit Knoten $\{\,a_1,a_2,a_3,a_4,a_5\,\} $ und Kanten
$\{\,(a_5,a_3),(a_4,a_5),(a_3,a_4),(a_2,a_1),(a_1,a_3),(a_1,a_2)\,\} $.

\begin{longtable}[c]{@{}ll@{}}
\toprule\addlinespace
Semantik & Extensionen
\\\addlinespace
\midrule\endhead
GR & $\{\,\{\,\,\}\,\} $
\\\addlinespace
CO & $\{\,\{\,\,\},\{\,a_2\,\},\{\,a_1,a_4\,\}\,\} $
\\\addlinespace
ST & $\{\,\{\,a_1,a_4\,\}\,\} $
\\\addlinespace
PR & $\{\,\{\,a_2\,\},\{\,a_1,a_4\,\}\,\} $
\\\addlinespace
\bottomrule
\end{longtable}

\section{Eingabe AF}\label{eingabe-af}

Gegeben sei ein AF

\[ F = (A,R) \]

mit

\[ A = \{a_1,a_2,\dots,a_n\} \]

einer Menge von $n$ Argumenten und

\[ R \subseteq A \times A \]

einer Angriffsrelation.

\subsection{Beispiel}\label{beispiel}

\subsubsection{alt}\label{alt}

Gegeben ist das AF

\[ F = (A,R) \]

mit

\[ A = \{a_1,a_2,a_3\} \]

und

\[ R = \{(a_1,a_2), (a_3,a_2), (a_1,a_1)\} .\]

\subsubsection{neu}\label{neu}

Gegeben ist das AF

\[ F = (A,R) \]

mit

\[ A = \{\,a_1,a_2,a_3,a_4,a_5\,\} \]

und

\[ R = \{\,(a_5,a_3),(a_4,a_5),(a_3,a_4),(a_2,a_1),(a_1,a_3),(a_1,a_2)\,\} .\]

\section{Problem}\label{problem}

Es soll ein CSP fuer \emph{EE-CO} erstellt werden. Dabei soll jede
Loesung des CSP einer Extension in $Cmpl(F)$ entsprechen.

\section{CSP}\label{csp}

Unser CSP soll $2n$ Variablen haben, also doppet so viele Variablen, wie
das AF Argumente hat.

Das CSP $\mathscr{P}$ ist ein Tripel

\[ \mathscr{P} = (X,D,C) \]

mit $X = \{x_1,x_2,\dots,x_{2n}\} $ einer Menge von Variablen,
$D = (D_{x_1},D_{x_2},\dots,D_{x_{2n}}) $ einem $2n$-Tupel endlicher
Domaenen und $C$ einer Menge von Constraints.

Die Variablen sind alle boolsche Variablen, also gilt fuer die
jeweiligen Domaenen
\[ D_{x_i} = \{true, false\} \quad\text{fuer alle}\quad 1\leq i \leq 2n .\]

\section{Variablen des CSP}\label{variablen-des-csp}

Die Variablen

\[ X  = \{x_1,x_2,\dots,x_n,x'_1,x'_2,\dots,x'_n\} = X_{\phi} \cup X_{\phi'} \]

des CSP setzen sich aus zwei Mengen \[X_{\phi} = \{x_1,x_2,\dots,x_n\}
\] und \[X_{\phi'} = \{x'_1,x'_2,\dots,x'_n\} \] zusammen.

Wir definieren zwei Bijektionen $\phi \colon A \to X_{\phi}$ und
$\phi' \colon A \to X_{\phi'}$ mit

\[ \phi(a_i) = x_i \quad \text{fuer} \quad 1 \leq i \leq  n \]

und

\[ \phi'(a_i) = x'_i \quad \text{fuer} \quad 1 \leq i \leq  n,\]

die jeweils jedem Argument in $A$ eineindeutig eine Variable in
$X_{\phi}$ bzw. $X_{\phi'}$ zuordnen. Die Variablenmengen $X_{\phi}$
bzw. $X_{\phi'}$ sind dann durch das Bild der jewiligen Abbildung
gegeben:

\[ X_{\phi}  = \phi(A) = \{ \phi(a) \,|\, a \in A \} \]
\[ X_{\phi'}  = \phi'(A) = \{ \phi'(a) \,|\, a \in A \} \]

\subsection{Beispiel}\label{beispiel-1}

In unserem Beispiel gilt

\[ X_{\phi} = \{x_1,x_2,x_3\} \] und \[ X_{\phi'} = \{x_1',x_2',x_3'\}\]
und daher \[ X = \{x_1,x_2,x_3,x_1',x_2',x_3'\}.\]

\section{Konfliktfreiheit}\label{konfliktfreiheit}

\subsection{Beispiel}\label{beispiel-2}

\[*constraints*\]

\begin{itemize}
\itemsep1pt\parskip0pt\parsep0pt
\item
  $\lnot ( A_5 \land A_3 ) $
\item
  $\lnot ( A_4 \land A_5 ) $
\item
  $\lnot ( A_3 \land A_4 ) $
\item
  $\lnot ( A_2 \land A_1 ) $
\item
  $\lnot ( A_1 \land A_3 ) $
\item
  $\lnot ( A_1 \land A_2 ) $
\end{itemize}
